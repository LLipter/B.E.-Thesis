% Chapter x

\chapter{安全性设计}

\section{前端页面的生成规则}

所有返回给用户的前端页面都经由Thymeleaf模板引擎渲染而成,我们遵照规范,正确的使用了Thymeleaf模板,因此从源头杜绝了所有可能的XSS攻击,包括Stored XSS和Reflected XSS漏洞都不会出现在我们的网站中。

\section{注册表单校验规则}\label{sec:registerV}

\begin{itemize}
	\item \textbf{用户名校验规则}
	
	为保证用户名的强度,我们要求一个合法的用户名必须不少于6个字符,并不多于18个字符。每一个字符都必须是ASCII字母,ASCII数字,或者下划线“\_”。
	
	\item \textbf{密码校验规则}
	
	密码的强度对于一个系统的安全来说至关重要,弱密码如“123456”、“root”、“qwert”都极易被攻击者猜中。因此为保证用户的密码强度,我们要求一个合法的密码必须必须不少于6个字符,并不多于18个字符。每一个字符都必须是可打印出来的ASCII字符。并且整个密码必须同时包含大写字母、小写字母、数字和特殊符号(例如百分号、下划线等)。
	
	\item \textbf{电子邮箱地址校验规则}
	
	我们认为电子邮箱地址应当是一个合法的互联网地址,因此利用了JavaMail中的InternetAddress类来校验它。代码示例如下。
	
	\begin{minted}
	[
		frame=lines,
		framesep=2mm,
		baselinestretch=1.2,
		bgcolor=lightgray,
		fontsize=\footnotesize,
		linenos
	]{java}

public static boolean isValidEmail(String email) {
        try {
                InternetAddress emailAddr = new InternetAddress(email);
                emailAddr.validate();
                return true;
        } catch (AddressException ex) {
                return false;
        }
}

	\end{minted}
	
	\item \textbf{重复密码校验规则}
	
	为确保用户输入了其设想中的密码,避免由于键盘误操作带来的错误。我们要求用户输入两次索要设置的密码,只有两次密码完全相同的时候才被认为密码正确,通过校验。
	
\end{itemize}

\section{电子邮箱的验证规则}\label{sec:emailV}

用户在刚刚注册完,或者是刚刚更换电子邮箱地址之后。这个新的电子邮箱地址会处于“未验证”的状态。我们估计用户通过我们的系统验证这个电子邮箱地址。当用户在登录后点击个人主页中的“VERIFY YOUR EMAIL ADDRESS”链接后,一封电子邮件会被发送到用户指定的邮箱中,其中会包含一个链接,点击该链接就可以完成对电子邮箱的验证。

\section{文件上传的验证规则}\label{sec:fileV}

文件的上传在网络生活中及其常见,应用广泛。但往往不严谨的实现会留下安全隐患,给恶意攻击者留下攻击服务器的空间。我们在这里进行了多次验证,主要防范了以下的潜在攻击方式
\begin{itemize}

	\item \textbf{耗尽服务器资源的攻击}
	
	为了防止服务器的磁盘资源被恶意攻击者用很多大文件抢占一空,我们限制了单个上传的文件大小上限,现阶段我们只允许上传小于10MB的文件。
	
	\item \textbf{恶意文件名攻击}
	
	恶意攻击者有可能会上传形如“../../file.txt”的文件,使得该文件有可能会存储在服务器指定的存储根目录外,造成未知的、及其难以排查的错误。因此我们会仔细检查文件名,拒绝所有包含路径分隔符的非法文件名。
	
	\item \textbf{恶意程序注入}
	
	恶意攻击者有可能会上传包含恶意程序的代码片段,并试图通过别的手段让这段代码在服务器端运行,从而制造后门或或获取更大的服务器权限。因此我们采取文件拓展名白名单的形式限制了可以上传的文件类型。只有形如以下拓展名的文件才能被成功上传:
	\begin{itemize}
		\item \textbf{文档类}
		
		docx,doc,pptx,ppt,xlsx,xls,pdf
		
		\item \textbf{图片类}
		
		jpeg,jpg,png,gif,tiff,tif,raw,bmp,svg
		
		\item \textbf{纯文本类}
		
		txt,tex,md
		
		\item \textbf{音频类}
		
		mp3,wav
		
		\item \textbf{压缩类}
		
		zip,z,rar,7z,gz,tar
		
		\item \textbf{数据类型}
		
		csv,xml
	\end{itemize}

\end{itemize}

