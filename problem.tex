\documentclass[]{WHUBachelor}

\begin{document}

\chapter*{答辩纪要}

2016302580055 - 了然 - 基于B/S模式的文件数字签名管理器的设计与实现

\subsection*{如何利用JavaMail模组发送邮件的?这部分的运作机制是怎样的?}

我们将通过JavaMail库的帮助与一个SMTP服务器建立联系,并最终将邮件推送到用户的设备上。需要用到邮件服务器地址、端口号、用户名、密码等信息。这部分信息可以写入某个配置文件。同时如果需要更大规模的电子邮件服务,可能还需要像第三方购买,包括邮件服务器。也可以自己开发。

\subsection*{为何在进行数字签名前要进行SHA256哈希?}

一是文件的长度不固定,直接签名可能在计算上效率低,二是直接对文件进行哈希签名可能会留下安全隐患,存在在某种特性场景下伪造数字签名的可能性。

\subsection*{程序主密钥在项目中起怎样的作用?是如何保存的?}

程序主密钥是一切加密解密、保证项目安全的核心。在设计中该密钥应该是完全随机生成的,并且在项目的运行过程中只存在于内存中,只在启动项目的时候需要载入密钥,其他时候密钥可以离线存储于安全的位置,比如断网放在物理上隔离开来的保险箱中。

\subsection*{用户的隐私是如何保护的?}

用户的密码只会存储密码哈希加盐过后的版本,即使是项目的后台管理者也不能获得用户的明文密码。但由于用户用于加密的密钥是由后台程序统一管理的,后台可以查看用户的加密文件。一个改进是,可以让加密密钥由用户的密码推导而出,但这样一来相当于把保证安全的指责从服务器端转移给了用户,用户必须时刻牢记自己的密码,一旦密码丢失,这部分加密过后的文件也会永久性的丢失。

\subsection*{如何保证用户的登录状态?}

我们将随机生成的Session放入用户的Cookie中,通过判断Session来确定用户,保持登录状态。


\end{document}